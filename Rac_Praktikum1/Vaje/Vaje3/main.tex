\documentclass{article}
\usepackage[utf8]{inputenc}
\usepackage{amsmath}
\usepackage{graphicx}

\title{Prvi skok v LaTeX}
\author{Vida Groznik}
\date{18. november 2021}

\begin{document}

\maketitle
\newpage

\tableofcontents
\listoftables
\listoffigures

\newpage

\section{Uvod}
Tole je besedilo, ki je del našega novega dokumenta. Besedilo lahko razdelimo na poglavja in podpoglavja, lahko ga oblikujemo, vstavljamo slike in tabele, lahko uporabljamo reference itd.

Besedilo lahko razdelimo tudi na odstavke.

\paragraph{Hipoteza 1:} Vsi ljudje znamo brati in pisati. 

\subparagraph{Podhipoteza 1.1} tole je nadaljevanje prejšnje hipoteze.

\section{Matematični izrazi}
Za pisanje matematičnih izrazov lahko uporabljamo različna okolja.....
\subsection{Uporaba okolja equation}
\subsubsection{Številčenje}
\begin{equation}
    f(x) = x^2
\end{equation}

\subsubsection{Brez številčenja}
\begin{equation*}
    g(x) = 3x^3 + x + 12
\end{equation*}

\subsection{Uporaba okolja align}
\begin{align*}
    1 + 2 &= 3 \\
    y &= 3x^3 - x3^2 +11x + 2 \\
    f(x) &= \frac{1}{x} \\
    g(x) &= \int^b_a \frac{1}{4}x^2 \\
    F(x) &= \sqrt{x}
\end{align*}

\subsection{Uporaba okolja matrix}

$
\left(
\begin{matrix}
1 & 1 & 0 \\
0 & 1 & 1 \\
0 & 0 & 1
\end{matrix}
\right)
$

\subsection{Funkcija v besedilu}
Matematični izraz lahko napišemo tudi v vrstici z besedilom $y = k*x + n$.

\section{Fotografije}
Na sliki \ref{fig:nemo} je predstavljen primerek te vrste.

\begin{figure}[h]
    \centering
    \includegraphics[width=\linewidth]{nemo.jpg}
    \caption{Riba Nemo.}
    \label{fig:nemo}
\end{figure}

\section{Tabele}
\label{sec:Tabela}
V tabeli \ref{tab:predavatelji} je sezam predavateljev.

\begin{table}[]
    \centering
    \begin{tabular}{l l|c}
        \hline
        \textbf{Ime} & \textbf{Priimek} & \textbf{Kabinet}\\
        \hline \hline
        Vida & Groznik & K1.03 \\
        Nino & Bašić & K1.09 \\
        Matjaž & Krnc & K1.08 \\
        \hline
    \end{tabular}
    \caption{Seznam predavateljev}
    \label{tab:predavatelji}
\end{table}

\section{Oblikovanje besedila}
\textbf{Na Slovaškem} \underline{od danes velja zaprtje javnega življenja} \textit{za necepljene.} Pravilo PC (preboleli in cepljeni proti covidu-19) velja za hotele, gostinske lokale in prireditve. Pogoj PC je za hotele, gostinske lokale, storitve in prireditve od danes v veljavi tudi na Češkem. \textsc{Poleg tega naj bi v podjetjih znova uveljavili hitro testiranje na okužbo z virusom SARS-CoV-2.}
\subsection{Blablabla}
\texttt{Na Slovaškem od danes velja zaprtje javnega življenja za necepljene.} \textsf{Pravilo PC (preboleli in cepljeni proti covidu-19)} velja za hotele, gostinske lokale in prireditve. Pogoj PC je za hotele, gostinske lokale, storitve in prireditve od danes v veljavi tudi na Češkem. Poleg tega naj bi v podjetjih znova uveljavili hitro testiranje na okužbo z virusom SARS-CoV-2.
\subsection{Blablabla verzija 2}
Na Slovaškem od danes velja zaprtje javnega življenja za necepljene. Pravilo PC (preboleli in cepljeni proti covidu-19) velja za hotele, gostinske lokale in prireditve. Pogoj PC je za hotele, gostinske lokale, storitve in prireditve od danes v veljavi tudi na Češkem. Poleg tega naj bi v podjetjih znova uveljavili hitro testiranje na okužbo z virusom SARS-CoV-2.
\end{document}
